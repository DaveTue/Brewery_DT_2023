Digital twins (DTs) are beneficial to the management of complex product life cycles because they can offer insights from both physical and virtual worlds. In order to reach its full potential, a DT often incorporates a broad range of models and toolsets. The integration and orchestration of these components pose a challenge to the DT developer. In this project, we develop a case study of a microbrewery DT and investigate frameworks that enable integration and orchestration techniques in order to simplify the design process for the trend of growing count of models in DTs.

A DT is the composite of elements from five dimensions, namely physical entities, virtual entities, services, data, and the interconnections between them. In a closed-loop, the virtual entities collect data from the physical entities, apply the data analysis services, and eventually optimize the physical entities. 

As DTs become increasingly complex, diverse tools and heterogeneous models inevitably must be brought together in order to cover all aspects of the systems. From there, the challenges in integration and orchestration arise. In short, integration concerns the encapsulation and the interface between virtual entity models. On the other hand, orchestration addresses the problem of execution sequences.
  
This study set out to investigate frameworks that embody techniques for integration and orchestration. The purpose was to identify good practices which could be reused in DT developments. A microbrewery DT was built as a case study. For now, it supports two services, namely Production Prediction and Production Control. We selected and applied three frameworks of distinctive styles to demonstrate the services. They were TwinOps---inspired by the DevOps principle; ThingsBoard---an open source platform based on the Internet of Things (IoT) and Service-Orientated Architecture (SOA); and finally Ptolemy II---based on an actor-oriented architecture which aims at experimentation in cyber-physical systems (CPS). We consider these three frameworks for their wide variety of uses, and presumed suitability for DTs.

The results indicate that a distributed framework architecture, such as TwinOps and ThingsBoard, enables a high degree of modularity and configuration which supports integration of diverse components. The extensive configuration options also contribute to a higher automation level for orchestration, as initial setup is paid off by automated processing during operations. In contrast, in a monolithic framework like Ptolemy II, the core functionalities have been unified in a self-contained package rather than assembled from a series of deployable modules or pipeline stages. As a result, it can benefit from smaller communication overhead in integration, which suggests better system timeliness. However, the same tight coupling characteristic may also restrict the flexibility of the orchestration.

Our findings highlight possible techniques for integration and orchestration. The comparison of the frameworks should be of guidance to developers considering extending a framework that suits their DT use cases.
 
